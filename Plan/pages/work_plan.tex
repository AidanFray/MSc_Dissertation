\chapter{Work Plan}

\section{Literature review}
\label{litReview}
\textbf{Duration}: ? 
\\\\
This stage will involve reviewing the research into producing near collisions with SHA1. This may also extend into producing a review around full collisions of SHA1.
\par
SHA1 has been selected due to its vulnerability. Methods for producing collisions have been known for a long time and it seems the algorithm is on its way out of mainstream use.

\section{Formulation specific research questions}
\textbf{Duration}: ?
\\\\
Questions may currently involve:

\begin{itemize}
    \item What is the probability of producing near-collisions with SHA1?
    \item What would be the complexity in storage and computation?
    \item Can such near-collisions be generated e.g. using the university’s high-performance computing facilities? 
\end{itemize}

Further questions may be added using information gained from already performed research.



\section{Implementing near-collision code}
\label{code}
\textbf{Duration}: ?
\\\\
This section will involve the actual implementation of code using knowledge gained through the literature review. This may be pre-existing code manipulated to work on the university’s HPC cluster.

\section{Determining metrics for measuring experiments}

Ways to measure the progress of a project

\section{Experimentation stage}

\textbf{Duration}: ?
\\\\
This stage will involve running experiments to gain complexity data. This may involve working on further research and validating results.

\section{Thesis/Paper writing}

\textbf{Duration}: ?
\\\\
This will involve the creation of a \textasciitilde70 page dissertation alongside a condensed paper for submission to a journal
