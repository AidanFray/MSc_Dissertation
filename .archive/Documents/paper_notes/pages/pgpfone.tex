\section{Whole-Word Phonetic Distances and the PGPfone Alphabet}

PGPfones designed solution was the development of a wordless based of military NATO codes.

Key terms: ``Phonetic confusability''

The authors used Moby Pronunciator database of nearly 200,000 word/pronunciation pairs.

Use-case for these findings are when verification needs to occur over a auditory line.

Words are created from ``phonemes'' (Unit of sound in a word). They state the the distance between two words can be approximated by between the phonemes that make up the word.

\textit{``It should be noted that this is only one of many possible approaches.''}

Mentions on potential further work in the emprical evaluation of these points

``A larger list (two
bytes per word) would require nearly 650 kilobytes of mem-
ory, as well as a word vocabulary larger than most speakers'
vocabulary.''

Big assumption that the two word PGP word list will be memorised by people ``This assumes, of course, that the (listening) human can tell from which list a word has been drawn.''. However, I think this was rectified with the 2-syllable then 3-syllable trick for even odd words.

Many questions proposed by the author:
\begin{itemize}
    \item Do the pronunciation assumptions fail when the reader is not a native English speaker?
    \item How confusable are the words?
\end{itemize}