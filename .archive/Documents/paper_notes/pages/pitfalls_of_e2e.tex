\section{On the Pitfalls of End-to-End Encrypted Communications: A Study of Remote Key-Fingerprint Verification}

Paper compares the fingerprint verification task in the context of Instant messaging E2E The paper is also choosing to focus on the remote vs proximity ways of comparing fingerprints. They aim to study the security and usability of human-centred E2E key verification.
The study ran with 25 participants.

Seems to be the first study of remote code verification study

Study compares Numerical, Image and verbally spoken codes and QR just for proximity alongside all the other alternatives.

The studies main aim to compare remote to proximity, not compare the various methods

The research is split into two sub-goals: Robustness - looks at the FAR and FRR. User experience and perception - System usability, Comfort, Satisfaction and Adaptability

States that numerical is the one of the most common representations, but out of the chosen methods 50\% use numerical and the rest hexadecimal?

They included 3 level of incorrect codes. One char change, one block and the entire fingerprint

Overall Audio was one of the most secure and easiest to use. However results for image based comparisons were not controlled in any way due to the author being unable to access the telegram image generation, this means while the other schemes had controlled restraints the image did not. Images was rated the worst.


Findings:
\begin{itemize}
    \item High False Acceptance Rate (False Negative) for all code verification methods 
    \item Results point to low usability in remote code verification settings, aside from audio based verification ever scheme has above 20\% False Rejection Rate (False Positive)
    \item In proximity settings users were able to detect attacks with minimal error rates
\end{itemize}