\section{Loud and Clear: Human-Verifiable Authentication Based on Audio}

Paper is in the context of secure device pairing. The paper proposes the use of Loud-and-Clear which uses text to speech to vocalise a robust english like sentence derived from a public key


"Stajano and Anderson proposed a method
for establishing keys by means of a link created through physical contact [9]."

``A number of efforts have been made to involve a human user in the secondary channel in order to manually verify/compare
keys (or hashes thereof) including [24, 11, 12] and [21].''

CHECK: Manual authentication for wireless devices.

``system represents a cryptographic hash as a series of six short (up to four-
letter) words'' - N. Haller. Rfc1760: The s/key one-time password system, 1995.

System truncates the hash and encodes into 10-bit sections.

The system is based of MadLib sentences that are used mainly as amusment for children where gaps in text are filled with nouns

The effort placed into making an order for the wordlist is similar to that of PGPfone. They effectively want to make sure that one bit change, results in a completely distinct word

The paper is not backed up by any empirical data and is just a technical description on the system. It included just a performance analysis.