\section{Can Unicorns Help Users Compare Crypto Key Fingerprints}

This paper tested 8 different fingerprint representations with 661 different participants. All of these representations are testing using compare-and-confirm (basic comparison process for fingerprints) and compare-and-select (select from a list)

\begin{itemize}
    \item Hexadecimal
    \item Alternating vowel/consonant
    \item Words
    \item Numbers
    \item Sentences
    \item OpenSSH visual host key
    \item Vash
    \item Unicorns
\end{itemize}

The paper also emphasises the attention that has been taken to realism in the experimentation stage.

Talks about long term usability issues with public-key crypto and states fingerprints are a main source of the problem. ([28])

There seems to be mixed findings with compare-and-select and compare-and-confirm ([14, 17])

There has been research into OpenSSH fingerprints and how users interact ([19, 24])

Paper chose to target a security level of 160-bits

Papers attack strengths were $2^{40}$, $2^{60}$ and $2^{80}$

Paper states that MTurk users have been shown to be younger and better educated than the general US population ([15])

\subsection{Findings}

\begin{itemize}
    \item Graphical representations have mixed success rates (with quick comparison times)
    \item Recommendations not to use \textbf{compare-and-select}
    \item \textbf{No method} really provided enough security for high risk situations, they recommend removing users from the loop by using smart phone cameras etc
\end{itemize}