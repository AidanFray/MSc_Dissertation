\section{A Study of User-Friendly Hash Comparison Schemes}

The aim of this study is to provide a study on what hash comparison scheme provides the best accuracy and shortest comparison time.

The paper also proposes new schemes: An extension to Flag and three asian character representations. 

An interesting aspect is the effort put into the study to link characteristics about the participants, i.e. gender or age to how well they perform or rate the usability of the scheme. In total they had 436 participants.

Overall they compare:
\begin{itemize}
    \item Base32
    \item English words
    \item Random Art
    \item Flag
    \item T-Flag
    \item Flag ext
    \item Chinese, Korean, Japanese characters
\end{itemize}

Key phrase: "We did not use hexadecimal digits because they're similar to Base32 and known to be error prone" [NO REFERENCE]. This is shown in \textit{``An Empirical Study of Textual Key-Fingerprint Representations''} that this isn't true - ``However, our work shows that numeric representations actually perform significantly better than Base32 and is less error prone.''

They study also considers the comparison of "easy" and "hard" pairs of hashes. Hard is where the hashes are designed to be very similar. This is designed to provide a base-line (easy) and a worst-case (hard).

The paper deals with ranges of entropy of 22-28bits.

Another system that uses word is the Unmanaged Internet Architecture [4]

States that users make errors when comparing long strings [11]

Dicussions are also made into the pre-requirements needed for some of these schemes. Random-Art needs a colour display and requires a relativly high computational cost to generate a representation. Asian characters require unicode or codec support.

Paper also touches on ``Additional Benefits'' - the paper talks about the ability to verbally describe the scheme, this is so the encoding scheme can be verified using a channel that does not involve a visual feed such as a telephone conversation

Overall the paper provides a table of requirements for the schemes with the graphical representations requiring a large number of pre-requirements.

\subsection{Findings}
\begin{itemize}
    \item Source, age and gender have no significant impact on the accuracy across the schemes. But younger participants were significantly faster
    \item In general, Base32, Random Art, T-Flag and Flag Ext provide fast and accurate comparisons
    \item Comprehension of a language when used as a encoding scheme affects the ability to distinguish hard pairs by a significant amount
\end{itemize}
