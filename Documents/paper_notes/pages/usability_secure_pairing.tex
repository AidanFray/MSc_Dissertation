\section{Usability Analysis of Secure Pairing Methods}

Paper is aiming to perform a comparative usability evaluation of \textbf{selection methods} of hash comparisons

This paper is doing this study in the context of secure device pairing in a diffie-hellman type key exchange. However, the results of the study could directly apply to that of hash comparison, however, it must be considered that the comparisons here are for \textbf{very} short strings. Around 4-digits 

Comparison methods compared:
\begin{itemize}
    \item Compare-and-Confirm
    \item Select-and-Confirm
    \item Copy-and-Confirm
\end{itemize}

The paper also compared to methods of passphrase comparison that weren't relevant.

Paper also included demographical information in the study, with two round of 40 participants in the usability studies.


Findings:

Round 1:
\begin{itemize}
    \item Copy-and-confirm was percieved as hard to use
    \item Copy-and-confirm has the lowest fatal error rate but the longest comparison time and lowest percieved usability
    \item The schemes reduced fatal error rate as their comparison times rose
    \item Copy-and-confirm had issues with user confusion in terms of use
\end{itemize}

In the second round they only decided to continue with

\begin{itemize}
    \item Compare-and-Confirm
    \item Select-and-Confirm
\end{itemize}

They dropped Copy-and-confirm due to its high similarity to the other scheme "Copy" that was being used for passphrase sharing

The second round included some UI changes that were backed up by external research.

This round the changes performed on the schemes resulted in much lower fatal error rates. However, Select-and-confirm had an unacceptable 5\% error rate. Compare-and-confirm had 0\% fatal error rate. The was an improvment from 20\% to 0\%.

Both schemes were again percieved as easy to use

Overall it was shown that compare-and-confirm was the superior choice.