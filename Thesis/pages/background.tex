\chapter{Background}
\label{cha:Background}

\section{Problem Context}
% Goal of the project? Why am I concentrating on PEP's Trustwords?
The project aims to asses the security of \pep's fingerprint representation "Trustwords". Trustwords aim to sacrifice security for increased usability. The encoding scheme has been designed to assist this by having  the user compare a reduced set of words by about half in comparison to other schemes. To achieve this a larger number of bits are required per word. In this case Trustwords maps 16bits per word providing 65537 different combinations. This is, however, the largest number of bits per words (As show in the Literature Review in Chapter \ref{cha:LiteratureReview}) and pushes the boundaries of a language's usable vocabulary.

This abnormal design choice has not received a security assessment alongside no research backing up claims. This paper, therefore, aims to rectify this research gap.

However, in order to fully understand the project goal it is important to understand public-key cryptography, the role and functioning of \pep and the role fingerprint comparison has in preventing Man-in-the-middle attacks.

\textbf{TODO:} Maybe research questions here instead of in the literature review

\section{Public-key Cryptography Key Exchange}
Asymmetric cryptography facilitates the secure encryption of messages in end-to-end encryption (E2EE), verification of digital signatures and sharing of pre-communication secrets, among others. The use-case of E2EE messages will be the primary focus of this paper. 

The asymmetry stems from the use of "Public" and "Private" keys. The Public key is used to encrypt data that only the respective Private key can decrypt. This means the keys required to communicate can be easily shared across insecure channels.

To construct an E2EE connection the initial stage will be a key exchange using the key types discussed previously. This is the sharing of a pre-shared secret between two verified parties, this will then be used to encrypt subsequent messages with symmetric encryption\footnote{This is due to the speed increase of symmetrical over asymmetrical encryption}. This, however, hinges on the verification of the initial parties, if one party impersonates another and gets to view the pre-shared secret all communication becomes decryptable. Therefore, the correct identification of parties is crucial to maintaining secure communication channels. The exploitation of this is known as a Man-in-the-middle (MiTM) attack. This can be bidirectional where the attacker can sit in the middle of a communication channel and decrypt all correspondence. Figure \ref{fig:mitm} contains a visual representation of this attack.

\begin{center}
    \input{diagrams/MITM/basic}
    \begin{figure}[h]
        \caption{Photo depicting a MITM attack}
        \label{fig:mitm}
    \end{figure}
\end{center}

Due to the assumption that the attacking party (Eve) does not have the same public-private key pair as either party (Alice or Bob) the fingerprint of the public key can be used for identification. There are, however, in linking of key fingerprint to real world entity. This paper will, however, assume that users know the respective parties' correct fingerprint, as the protocols behind this are outside the project's scope. 

The fingerprint of the key is generated by running important sections of the key through a secure one way hash function. This process produces a digest of fixed length that can be used to compare keys. Therefore, the comparison of expected and actual fingerprints is used to detect MiTM attacks and prevent impersonation.

\section{PGP}
Pretty Good Privacy (PGP) is a specific implementation of public key cryptography discussed in the previous section. Its main use-case is for securing message based communications but can also be used for file or hard drive encryption.
\textbf{[...]}

\section{Pretty Easy Privacy/Trustwords}

% Does this need explaining?
\section{OpenCL}