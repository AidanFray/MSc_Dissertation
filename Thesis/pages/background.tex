\chapter{Background}
\label{cha:Background}

\section{Problem Context}
% Goal of the project? Why am I concentrating on PEP's Trustwords?
The project aims to asses the security of \pep's fingerprint representation "Trustwords". Trustwords aim to sacrifice security for increased usability. The encoding scheme has been designed to assist this by having  the user compare a reduced set of words by about half in comparison to other schemes. To achieve this a larger number of bits are required per word. In this case Trustwords maps 16bits per word providing 65537 different combinations. This is, however, the largest number of bits per words (As show in the Literature Review in Chapter \ref{cha:LiteratureReview}) and pushes the boundaries of a language's usable vocabulary.

This abnormal design choice has not received a security assessment to back up claims. This paper, therefore, aims to rectify this research gap.

However, in order to fully understand the project goal it is important to understand public-key cryptography, the role and functioning of \pep and the role fingerprint comparison has in preventing Man-in-the-middle attacks.

\textbf{TODO:} Maybe research questions here instead of in the literature review

\section{PGP}
Pretty Good Privacy (PGP) is a specific implementation of public key cryptography discussed in the previous section. Its main use-case is for securing message based communications but can also be used for file or hard drive encryption.

Each element of a PGP key is split into what is know as a 'packet'. The 'fingerprint' packet v4 contains a version number, timestamp and the main elements of the keys algorithm (i.e. RSA exponent and modulus) formatted as MPIs\footnote{Multiprecision integers are unsigned integers used to hold large integers such as the ones used in cryptographic calculations.}. This data is then preceded with PGP packet header (used to specify length and version numbers). This is all then utilized as input into the one-way hash function SHA-1 to produce a 160-bit digest.

\textbf{TODO: } Generate an image of PGP v4 fingerprint packet

Each key pair is the exclusive-or of each key's fingerprint, this combined values is then broken down into 16-bit chunks and mapped to words via a pre-defined dictionary.

\begin{figure}[h!]
    \centering
    $KeyFingerprint_{1} \oplus KeyFingerprint_{2} = TrustwordsFingerprint$
    \caption{Creation of the combined Trustword fingerprint}
\end{figure}

\begin{wrapfigure}{r}{5cm}
    \centering
    \begin{BVerbatim}
    [...]
52127 ZYGOTE
52128 ZYGOTIC
52129 ZYMURGY
52130 AACHEN
52131 AARDVARK
52132 AAREN
    [...]
    \end{BVerbatim}
    \caption{Re-mapping position in Trustword dictionary}
    \label{fig:remap}
\end{wrapfigure}

Due to the number of words required ($2^{16}$) alongside the design choice to exclude slang and profanities the english Trustword dictionary requires dual-mapping of a section of words. Approximately 13633/65536 (20.8\%) of words are re-mapped in the dictionary leaving 51903 unique words. The re-mapping is also done on a loop with it remaining alphabetical. Figure \ref{fig:remap} shows the position in the dictionary where this occurs. This predictability within the dictionary will be explored later in the paper.




% Does this need explaining?
\section{OpenCL}

\section{Similarity metrics (Soundex etc)}