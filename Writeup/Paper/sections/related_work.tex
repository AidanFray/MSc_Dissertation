\section{Related Work}

\subsection{Encoding Schemes}
Encoding schemes are the physical method of encoding used to represent the fingerprint. The most common representation is alphanumerical with either \textbf{Hexadecimal} (0-9/A-F) or \textbf{Base32} (2-7/A-Z). These encoding schemes are the most popular due to their intuitive simplicity and lack of hardware requirements.

Fingerprints can also be encoded using natural language. The fingerprint can be chunked and mapped to a set of words. The same principle can be used to fill placeholders in a pre-defined sentence allowing the simple formation of syntactically correct English sentences. Moreover, other languages can be used such as Chinese, Japanese or Korean, to map fingerprint chunks to characters.

A substantial amount of work has been performed assessing the performance of encoding schemes. Results from this literature consistently show the effectiveness of language-based encodings such as Words or Sentences with accuracies ranging up from 94\% \cite{dechand2016empirical}\cite{tan2017can}\cite{kainda2009usability}. In all cases, these were the best schemes from the sets assessed. The exception to this is the work performed by \textbf{H. Hsiao \textit{et al.}}\cite{hsiao2009study} with Words achieving an abnormal accuracy of 63\%. All of the papers assessed simulated attacks and had minimal consideration for technical details regarding their execution. 

Further research has aimed at creating or improving language based encoding schemes. Work by \textbf{Juola} and \textbf{Zimmermann} \cite{juola1996whole} aimed to produce a word list where phonetic distinctiveness was prioritised alongside, \textbf{Michael Rogers'}\footnote{\url{https://github.com/akwizgran/basic-english}} research detailing a process to map fingerprints to pseudo-random poems.

To our knowledge no research has been performed investigating the unique design choices present in Trustwords. This, therefore, is a promising candidate for further research.