\emph{Abstract} -- Many encrypted connections require the comparison of a fingerprint to
protect against eavesdropping. A substantial amount of past work has
aimed to propose fingerprint representations that work better with hu-
man limitations. ``Trustwords'' proposed by pEp is an example of such a
scheme, where the fingerprint is encoded as words in an attempt to improve
usability.
\\\\
This works main aim is to assess if Trustword’s recommended minimum
number of four words is sufficient. In order to achieve this goal, this work
implements an attack on Trustwords and quantifies its effectiveness on
more than 400 participants. A tool called GreenOnion was designed to
assist in quantifying attack feasibility. GreenOnion improved substantially
on a similar tool’s ability to search for matches concurrently where it was
able to search for over 1.5 million concurrent keys. Our findings show a
substantial increase in attack success compared to related literature. We
believe this increase is due to levering the design flaws in the Trustword
scheme. Therefore, we believe a minimum of four Trustwords is insufficient
to provide even a basic level of security.