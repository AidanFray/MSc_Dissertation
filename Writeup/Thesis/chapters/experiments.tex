\chapter{Experimentation Results}
\label{cha:Experiments}

\section{Scallion vs GreenOnion}
\label{sec:SvG}

This section compares Scallion and the newly designed GreenOnion ability to search for a large number of potential keys.

Figure \ref{tab:scallion_speed} shows the speed decline as the number of concurrently checked potential keys increases. Alongside the results for base version of Scallion is the speed for \textit{Scallion Improved}. This is the same Scallion code but with a fix that reduces the severity of the speed decline.

\begin{table}[h!]
    \centering
    \begin{tabular}{|ll|}
        \hline
        \textbf{Scallion} & \\
        \hline
        OS & Windows 10 (May 2019) \\
        Version & Scallion 2.1 (Commit: \verb|42b4cb5|\footnote{\url{https://github.com/lachesis/scallion/tree/42b4cb555dbf8554b212a2bc3e0ba3d652434ecf}}) \\
        \hline
        \textbf{GreenOnion} & \\
        \hline
        OS  & Linux Mint 19.1 \\
            & Kernel: 4.15.0 \\ 
        \hline
        \textbf{Hardware} & \\
        \hline
        GPU & Nvidia GTX 750 Ti  \\
        CPU & AMD FX-6300 \\
        \hline
    \end{tabular}
    \caption{Testing environment}
\end{table}

Figure \ref{fig:scallionCode} contains a snippet of code with the error. The calls to \verb|parms.ToolConfig.PredictRuntime()| become increasingly expensive as the number of concurrently checked keys increases. Removing this line from the \verb|Console.Write| statement results in a sizeable improvement in speeds as can be seen in Figure \ref{tab:scallion_speed}. This was necessary to include as we believe the improved versions show a better comparison between the design of the two tools. 

\begin{figure}[h!]
    \begin{adjustbox}{width=\textwidth,height=15cm,keepaspectratio}
\begin{Code}[CSharpStyle]
    CConsole.Write("LoopIteration:{0}  HashCount:{1:0.00}MH  Speed:{2:0.0}MH/s  Runtime:{3}  Predicted:{4}", 
    loop, 
    hashes / 1000000.0d, 
    hashes/gpu_runtime_sw.ElapsedMilliseconds/1000.0d, 
    gpu_runtime_sw.Elapsed.ToString().Split('.')[0], 
    parms.ToolConfig.PredictRuntime(hashes * 1000/gpu_runtime_sw.ElapsedMilliseconds)
);
\end{Code}
\end{adjustbox}
    \caption{Scallion's main print statement}
    \label{fig:scallionCode}
\end{figure}

GreenOnion's speed is almost perfectly consistent. It is slower up until  around 800 keys. This initial lower speed is due to the overhead of the bloom filter; however, as the number of keys increase, the bloom filter's benefits become apparent. In this test, the bloom filter size was kept consistent at exactly 10,000 elements. 

Scallion was unable to handle anything over 2513 concurrent keys. This was due to the search strings being passed via a command line argument as a regex string. This number of keys hit the character limit of a Powershell command. GreenOnion, however, has been tested to handle more than 1.5 million keys with a slight reduction in performance. This improvement over Scallion is, therefore, substantial.

GreenOnion's code has been made publicly available and can be viewed on GitHub\footnote{\url{https://github.com/AidanFray/GreenOnion}}.

\begin{figure}[h!]
    \centering
    
\begin{tikzpicture}[every axis plot/.append style={thick}]
    \begin{axis}[
        width=\linewidth,
        height=10cm,
        grid=major,
        xmin=1, xmax=40,
        % ymin=0, ymax=100,
        xlabel=Number of Keys,
        ylabel=Hashing speed (MH/s)
        ]
        \addplot table [x index=0, y index=1, mark=none, search path=data, col sep=comma]{scallion.csv};
        \addlegendentry{Scallion}
        
        \addplot table [x index=0, y index=1, mark=none, search path=data, col sep=comma]{green.csv};
        \addlegendentry{GreenOnion}	
     \end{axis}
\end{tikzpicture}
    \label{tab:scallion_speed}
    \caption{Speed comparison between Scallion and GreenOnion}
\end{figure}

\newpage

\section{Experiment 1 - Metric performance}

As discussed in Section \ref{exp:metric}, the goal of the experiment was to select a set of metrics to be assessed in the following experiment. This section will discuss the demographics of participants alongside the subsequent results.

\begin{table}[h!]
    \centering
    \begin{tabular}{|l|ll|}
        \hline
        Gender & Male: & 44.4\% \\
               & Female: & 55.6\% \\
        \hline
        Age:   & 18-24: & 10.1\% \\
               & 25-29: & 20.2\% \\
               & 30-39: & 31.3\% \\
               & 40-49: & 21.2\% \\
               & 50-59: & 12.1\% \\
               & 60-69: & ~~4.0\% \\ 
               & 70-79: & ~~1.0\% \\ 
               
        \hline
        Highest Education:  
        & GCSE:                 & 13.1\% \\
        & A-Level/O-Level:      & 19.2\% \\
        & Bachelor's degree:    & 52.5\% \\
        & Master's degree:      & 13.1\% \\ 
        & PhD:                  & ~~2.0\%  \\
        \hline

    \end{tabular}
    \caption{Participant demographics}
    \label{tab:exp1_demo}
\end{table}

Overall, 104 participants were assessed in this study. Five results were discarded from the set due to either failing the attention questions (See in Section \ref{sec:exp1_qualitycontrol}) or having too low of a fluency rating. This dismissal was a necessary process to improve the health of the results. 

Table \ref{tab:exp1_demo} contains the demographical breakdown of the reduced set of participants. The average ages of the participants were 37.3 years ($\sigma = 11.67$) with a split of 44.4\% of Males to 55.6\% of Females. As can be seen, over 60\% of participants can be considered highly educated (Bachelor’s and up). This level of education is not sufficiently reflective of the general population and therefore, has to be considered when interpreting the results. All participants were sourced from the US; this again requires consideration due to the broad range of dialects present that may bias the results. Further work could investigate the effect of location and dialect on similar results. 

\begin{wraptable}[11]{r}{8cm}
    \centering
    \begin{tabular}{|l|l|l|}
        \hline
        \textbf{Metric} & \textbf{Average Rating}  & \textbf{$\sigma$}\\
        \hline
        Leven     & 3.66  & 1.15\\
        NYSIIS    & 2.92 & 1.31\\
        Metaphone & 2.56 & 1.32\\
        Phonetic Vec & 2.50 & 1.35\\
        Soundex & 2.08 & 1.12 \\
        \hline
        Random  & 1.16 & 0.46\\
        \hline
    \end{tabular}
    \caption{Average metric performance}
    \label{tab:exp1_results}
\end{wraptable}


Figure \ref{tab:exp1_results} shows the average results for the selected matches explained in Section \ref{sec:exp1_comparison}. It can be seen that Levenshtein came out substantially above the rest. The breakdown of the ratings in Figure \ref{fig:exp1_breakdown} also shows Levenshtein's dominance. Levenshtein has a much more significant proportion of 4 and 5 ratings than the alternatives. This performance may, however, be due to the visual way the comparisons are performed. (Discussed in detail in Section \ref{sec:exp1_considerations}). 

Due to the averages of Metaphone and Phonetic Vectors being so close standard deviation was used as the final decider. As can be seen, Megaphone has a slightly lower $\sigma$ value of that of Phonetic vectors, thus, contributing to the decision to select Metaphone.

\subsection*{Considerations}
As discussed in Section \ref{sec:exp1_considerations}, compromises were made to fulfil all requirement with a minimized cost. Levenshtein's abnormal performance further backs up concerns around the visual aspect biasing the performance of the metric. This aspect requires consideration when interpreting the results. 

Even with the discussed issues, the experiment was designed to reduce the number of metrics promptly. This experiment, therefore, has achieved that goal of providing three metrics for the subsequent experiment while balancing between accuracy and expenditure. The results are also an improvement over informal ad-hoc methods. Further work could aim to reproduce this study with the proposed audio-based design (See Section \ref{sec:exp1_considerations}).

\begin{figure}[h!]
  \minipage{0.5\textwidth}
    \begin{filecontents}{soundex.csv}
    rating,occurrence
    1,192
    2,150
    3,81
    4,56
    5,13
\end{filecontents}

\tally{soundex.csv}
    
    \caption{Soundex}
  \endminipage
  \minipage{0.5\textwidth}
    \begin{filecontents}{leven.csv}
    rating,occurrence
    1,30
    2,57
    3,91
    4,192
    5,125
\end{filecontents}

\tally{leven.csv}
    \caption{Levenshtein}
  \endminipage
  \\
  \minipage{0.5\textwidth}
    \begin{filecontents}{nysiis.csv}
    rating,occurrence
    1,90
    2,99
    3,128
    4,102
    5,69
\end{filecontents}

\tally{nysiis.csv}
    \caption{NYSIIS}
  \endminipage
  \minipage{0.5\textwidth}
    \begin{filecontents}{metaphone.csv}
    rating,occurrence
    1,146
    2,103
    3,103
    4,99
    5,40
\end{filecontents}

\tally{metaphone.csv}
    \caption{Metaphone}
  \endminipage
  \\
  \centering
  \minipage{0.5\textwidth}
    \begin{filecontents}{wordvec.csv}
    rating,occurrence
    1,167
    2,88
    3,104
    4,91
    5,42
\end{filecontents}
    
\tally{wordvec.csv}
    \caption{Phonetic vector}
  \endminipage
  \caption{Individual breakdown of results for each metric}
  \label{fig:exp1_breakdown}
\end{figure}

\newpage

\section{Experiment 2 - Trustword attacks}
\label{sec:exp2}
This experiment aims to quantify the success rate of the proposed attack. Design details for this experiment can be seen in Section \ref{sec:exp2_design}. This section presents and discusses the results of the experiment alongside a comparison to relevant literature.

\subsection*{Demographics}
Overall, 435 paid participants recruited via Amazon's MTurk were assessed in this experiment. We excluded 66 results; 7 due to being non-native speakers and 59 were discarded for failing the attention metrics (Discussed in Section \ref{sec:exp2_quality}).

\begin{table}[h]
    \centering
    \begin{tabular}{|l|ll|}
        \hline
        Gender & Male: & 50.4\% \\
               & Female: & 49.6\% \\
        \hline
        Age:   & 18-24: & 12.7\% \\ 
               & 25-29: & 18.2\% \\ 
               & 30-39: & 37.4\% \\ 
               & 40-49: & 17.9\% \\ 
               & 50-59: & ~~8.7\% \\ 
               & 60-69: & ~~4.3\% \\ 
               & 70-79: & ~~0.8\% \\ 

        \hline
        Highest Education:  
        & GCSE:                 & 13.8\%  \\
        & A-Level/O-Level:      & 24.1\% \\
        & Bachelor's degree:    & 51.5\% \\
        & Master's degree:      & ~~8.4\% \\ 
        & PhD:                  & ~~2.2\% \\
        \hline

    \end{tabular}
    \caption{Participant demographics}
    \label{tab:exp2_demo}
\end{table}

The reduced set of 369 participants had an average age of 36.63 ($\sigma = 11.35$) and consisted of an almost equal split of Male (50.4\%) to Female (49.6\%). Around 62\% of participants had completed a single stage of university (Bachelor and up). This aspect makes this set of participants more educated than the general population. All participants were also sourced from the USA and have rated themselves as fully native English speakers.
\begin{table}[!h]
    \centering
    \begin{tabular}{|l|l|}
        \hline
        Browser & Percentage \\
        \hline
        Chrome & 85.1\% \\
        Firefox & 12.2\% \\
        Safari & ~~1.9\% \\
        Other & ~~0.8\% \\
        \hline
    \end{tabular}
    \caption{Internet Browser breakdown}
    \label{tab:exp2_browser}
    \begin{tabular}{|l|l|}
        \hline
        Operating System & Percentage \\
        \hline
        
        \hline
        Windows (All) & 74.3\% \\
        ~~~~\textit{Windows 7} & ~~~12.7\% \\
        ~~~~\textit{Windows 8} & ~~~~~4.1\% \\
        ~~~~\textit{Windows 10} & ~~~57.5\% \\
        MacOS X	 & 13.8\% \\
        ChromeOS & ~~3.8\% \\
        Android	& ~~4.3\% \\
        iPhone	& ~~1.1\% \\
        iPad	& ~~0.5\% \\
        Linux	& ~~2.2\% \\
        \hline
    \end{tabular}
    \caption{Operating System breakdown}
    \label{tab:exp2_os}
\end{table}
    
As user-agent strings were collected for each participant collations of OS and internet browser versions was made possible. Tables \ref{tab:exp2_browser} and \ref{tab:exp2_os} contain the browser and OS breakdowns,  respectively. Chrome and Windows 10 seem to dominate the proportion of their respective domains alongside the surprisingly large proportion of workers utilizing older versions of Windows such as 7 or 8 (15.81\%).

\subsection*{Results}
Table \ref{tab:exp2_attacks} contains the break down of results for the experiment. It can be seen that the best metric out of the set was Levenshtein with an overall success of 19.80\%. The best performer, when regarding attack strength, as expected, is the \XOOX~attack. Levenshtein's \XOOX~attack performed the best overall with a success rate of 32.05\%. The worst performer was Metaphone, with an average of 16.88\% over its three levels of attacks. When comparing the performance of the metrics to the previous experiment the ordering remains the same with Levenshtein, NYSIIS and Metaphone all ranking in the same order.

\begin{table}
    \makebox[\textwidth][c]{
        \centering
        \begin{tabular}{|ll|l|l|l|}
            \hline
            \textbf{Metric} & & \textbf{Successful Attacks} & \textbf{Total Attacks} & \textbf{Success Rate} \\
            \hline
            Levenshtein && 218 & 1101 & 19.8\% \\
            \hline
            & \OOOO   & 34  & 358  & ~~9.5\% \\
            & \XOOO   & 59  & 353  & 16.7\% \\
            & \XOOX   & 125  & 390  & 32.1\% \\
            \hline\hline
            Metaphone &&  181 & 1072 & 16.9\% \\
            \hline
            & \OOOO   & 38 & 345 & 11.0\% \\
            & \XOOO   & 57 & 375 & 15.2\% \\
            & \XOOX   & 86 & 352 & 24.4\% \\
            \hline\hline
            NYSIIS &&  209 & 1114 & 18.8\% \\
            \hline
            & \OOOO   & 36 & 385 & ~~9.3\% \\
            & \XOOO   & 72 & 375 & 19.2\% \\
            & \XOOX   & 101 & 354 & 28.5\% \\
            \hline\hline
            \textbf{Overall} & & 608 & 3287 & 18.5\% \\
            \hline\hline
        \end{tabular}
    }
    \caption{Success rates for simulated attacks}
    \label{tab:exp2_attacks}
\end{table}

\subsection*{Considerations}
As previously mentioned, the set of recruited participants are more highly educated than the general population. This combined with the lower average age, contributes to an inevitable bias being introduced into the data. An assumption can be made that higher education leads to more skilful interaction with computer-based systems, this, therefore, with the previous assumption in mind means the results of this experiment can be considered a `best case' scenario. The attacks, therefore, would be expected to be more successful on less technical users. This consideration, however, requires further research and possible empirical data to determine the validity of the assumption. 

During this study, the main task was to perform the authentication ceremony a set number of times. In the real-world, the ceremony is a secondary task due to it not strictly being required. This aspect, alongside the potential for users to downplay the possibility of attack, leads to the conclusion that the success rates could be much higher in a real-world setting. The assumption is that these results display the success rate for attentive users where they make sure to double-check with their authentication partner multiple times. It is, therefore, assumed that the majority of users aim to complete the task as quickly as possible, if at all.

\subsection*{Comparison to alternative literature}
This section compares the results of the experiment to similar literature.

Work by \textbf{R. Kainda, I. Flechais} and \textbf{A. Roscoe}\cite{kainda2009usability} (previously discussed in Chapter \ref{cha:LiteratureReview}) is the most comparable to this experiment. They were the only study to use exactly 4 words but differed on the size of the dictionary (1024 words) and how a near-match were calculated. Near-matches were a difference in a single word, but without the consideration for similarity, this is the unique aspect considered with this experiment.

However, with those aspects in mind attacks on words encoding received an overall success rating of 3.3\%, considerably lower than that of this study. In the worst-case, our simulated attacks achieved almost 3 times as many successes compared to this study.

Other relevant work by \textbf{S. Dechand \textit{el al.}}\cite{dechand2016empirical} also discussed in Chapter \ref{cha:LiteratureReview} assess the success rate of attacks on words. Their experiment is less similar as 14 words per attack were assessed with each participant, where the comparison was performed visually. However, the success was 8.78\%. As 10/14 words were kept static, this result is more comparable to the highest attack strength ``2 static''. This, therefore, leads to another 3 fold improvement in the success rates of our simulated attacks. 

The final relevant paper to compare is that of \textbf{J. Tan \textit{et al.}}\cite{tan2017can}. This paper had the highest number of words assessed with 16 overall. With it assessing the same wordlist and attack strength as the work by \textbf{S. Dechand \textit{et al.}}\cite{dechand2016empirical}, the attack success was 14\% overall. This success rate compared to the highest attack assessed in the experiment results in another sizeable difference.

In conclusion, all related literature had much lower attack success rates that the results presented in this experiment. With all papers using similarly designed wordslists, it highly suggests that the deficiencies in Trustwords are the cause for the substantial increase in attack success. Alongside this, the consideration for similarity when calculating near-collisions could have also resulted in the higher attack success rates. This aspect is also a unique consideration compared to the available literature.

\section{Average number of near-collision keys for each metric}
\label{sec:averagePerms}
In order to predict the effectiveness of the attack over a wide variety of keys, the number of average near-collision keys requires computation. Table \ref{tab:average_perm} shows the average number of potential collision keys for each metric. The ``Compute Time'' is the computation time required for a single mid-range (2000MH/s) GPU.

\begin{table}[!h]
    \makebox[\textwidth][c]{
        \centering
        \begin{tabular}{|llll|}
            \hline
            \textbf{Metric} & & \textbf{Mean matches} & \textbf{Compute Time} \\
            \hline
            NYSIIS &&& \\
            \hline
            & \OOOO & 1963.07 &  ~~27.19 days \\
            & \XOOO & 448.37 &  119.14 days \\
            & \XOOX & 98.07 &  ~~~~1.49 years \\
            \hline\hline

            Metaphone &&& \\
            \hline
            & \OOOO & 27138.78 &  ~~~~1.97 days \\
            & \XOOO & 2822.74 &  ~~18.91 days \\
            & \XOOX & 339.84 &  157.45 days \\
            \hline\hline

            Levenshtein &&& \\
            \hline
            & \OOOO & 293.92 &  182.17 days \\
            & \XOOO & 102.75 &  ~~~~1.43 years \\
            & \XOOX & 35.53 &   ~~~~4.18 years \\
            \hline\hline

        \end{tabular}
    }
    \caption{Each metric's average number of near-collision keys and average compute times}
    \label{tab:average_perm}
\end{table}

As can be seen, metrics like Levenshtein have a low overall average and would require much more attacker resources to become feasible. On the other hand, Metaphone has attacks that could be computed on average in under 2 days. The success rate for this attack type in the second experiment was 11.01\% on average. This means on commodity hardware commonly found in the home; an attacker could compute a key in less than two days that could succeed more than 10\% of the time. This aspect is substantial and highlights weaknesses in the Trustword system.

\section{Distribution of vulnerable keys}
\label{sec:vulnKeys}
As discussed in the design of the second experiment (See Section \ref{sec:exp2}), the attacks sampled from a list of `vulnerable keys'. These vulnerable keys were created using real keys extracted from PGP key servers. These keys were then randomly sampled and used to create the combined key,as discussed in Section \ref{sec:pep}. 100,000 random key pairs were created and used as the sample\cite{davidnorman_2019}. Any vulnerable keys in this set were recorded and used in the experiment.

\begin{table}[!h]
    \centering

    \makebox[\textwidth][c]{

        \begin{tabular}{|lll|}
            \hline
            \textbf{Attack Type} & \textbf{Metric} & \textbf{Vulnerable \%} \\
            \hline 
            \XOOX~(152) && \\ 
            & Levenshtein & ~~4.295\% \\
            & Metaphone & 24.095\% \\
            & NYSIIS & 10.441\% \\
            \hline\hline
            \XOOO~(1525) && \\ 
            & Levenshtein & ~~0.827\% \\
            & Metaphone & 15.569\% \\
            & NYSIIS & ~~4.341\% \\
            \hline\hline
            \OOOO~(15250) && \\ 
            & Levenshtein & ~~0.157\% \\
            & Metaphone & 10.202\% \\
            & NYSIIS & ~~1.824\% \\
            \hline\hline
        \end{tabular}
    }
    \caption{Number of vulnerable keys per metric}
    \label{tab:vulnkeys}
\end{table}

Table \ref{tab:vulnkeys} contains the breakdown of the number of vulnerable keys. The takeaways from this table are the extremely low values for Levenshtein in the higher attacks. This phenomenon is due to the low number of overall matches produced by Levenshtein, meaning the metric is not effective in the vast majority of cases. However, the standout metric is Metaphone with a sizeable proportion of vulnerable keys; it was the worst performer when assessed against real users but only by as little as 2\%. If you take into consideration, the number of vulnerable keys detected in the set alongside the average amount of near-collision matches in Table \ref{tab:average_perm} Metaphone appears to be the most usable metric overall.

\section{Generation of keys}
\label{sec:key_gen}
In order to demonstrate that the generation of the near-collision keys is not only theoretical, the actual generation was performed using GreenOnion. A random metric was selected from the three carried forward from the first experiment. Alongside this, a random uncontrolled key was generated as the simulated attack target (See Appendix \ref{appendix:uncontrolled_key} for the armor public key). Then for each attack level (\OOOO, \XOOO, \XOOX), a key was computed. NYSIIS was chosen as the selected metric. To expedite the process, an initial search was performed for controlled keys that had the highest number of near-collisions; this was due to the demonstrative purpose of this experiment. Producing near-matching keys from key pairs with a very small number of potential matches adds very little to this demonstration. This, however, will result in best-case scenario key computation times.

\begin{table}[h!]

    \makebox[\textwidth][c]{
        \begin{tabular}{|l|l|}
            \hline
            Potential near-collision keys & 142,296 \\
            Hashing speed & 4000 MH/s \\
            Estimated computation time & 9 hours \\
            Actual computation & 6.41 hours \\
            GPU Days & 0.52 \\
            \hline
            Original Trustwords & BELL GRIND ALGERIA ANNULI \\
            Actual Trustwords   & BOIL GRAND ALGER ANNUL \\
            \hline
        \end{tabular}
    }
    \caption{NYSIIS - OOOO}
    \label{tab:nysiis0}

    \makebox[\textwidth][c]{    
        \begin{tabular}{|l|l|}
            \hline
            Potential near-collision keys & 133,200 \\
            Hashing speed & 8000 MH/s \\
            Estimated computation time & 4.2 hours \\
            Actual computation & 5.12 hours \\
            GPU Days & 0.853 \\
            \hline
            Original Trustwords & ASSUMING SONOMA DENS KEENER \\
            Actual Trustwords   & ASSUMING SUMMON DENNI CONNOR \\
            \hline
        \end{tabular}
    }
    \caption{NYSIIS - XOOO}
    \label{tab:nysiis1}

    \makebox[\textwidth][c]{
        \begin{tabular}{|l|l|}
            \hline
            Potential near-collision keys & 16,464 \\
            Hashing speed & 6000 MH/s \\
            Estimated computation time & 2.16 days \\
            Actual computation & 23.44 hours \\
            GPU Days & 2.93 \\
            \hline
            Original Trustwords & VOCATION BORE TANN ANTE \\
            Actual Trustwords   & VOCATION BARE TONE ANTE \\
            \hline
        \end{tabular}
    }
    \caption{NYSIIS - XOOX}
    \label{tab:nysiis2}
\end{table}

Tables \ref{tab:nysiis0}, \ref{tab:nysiis1} and \ref{tab:nysiis2} contain the parameters and results on the computation. Due to the varying availability of resources, each computation has a varying hashing speed. This variation has been normalized by the introduction of the "GPU Day" metric. A GPU day is simply the computation time required to reach the same conclusion on a single 2000 MH/s GPU. 

As can be seen most notably from Table \ref{tab:nysiis2}, the generation of the actual Trustwords differs in only 3 characters making it arguably very similar. When linking this generated key to results of the second experiment (See Section \ref{sec:exp2}) computation of 2.53 days on a single mid-range GPU can achieve around a 28\% success rate when assessed against actual users. Even with the least restricted attack type presented in Table \ref{tab:nysiis0}, the computation of half a day can provide almost a 10\% success rate when attacking users. This results in any user with a single commodity GPU could compute enough keys in less than a week to have a very high probability of deceiving an actual user.
