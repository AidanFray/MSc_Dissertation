\chapter{Project Aims}

This chapter defines the selected gaps and the subsequent research questions extracted.

The chosen project aims to asses the security of \pep's minimum recommendation of four Trustwords (As stated in Figure \ref{fig:trustwordsNum}). As discussed in the previous chapter, Trustwords aims to sacrifice security for increased usability. The encoding scheme has been designed to assist this by having  the user compare a reduced set of words. Moreover, issues with the dictionary's design, such as the presence of homophones and dual-mapping of words show the potential for possible vulnerabilities. 

As discussed in the previous chapter, the identified research gap regarding the minimal consideration of attack complexity would be a suitable addition for this project. This choice is due to it providing the ability to asses actual performance and, thus, the actual real-world strength of Trustwords while providing research to a scant area. Therefore, the security assessment is supplemented with complexity considerations alongside an actual implementation of the attack proposed.

\begin{center}
\begin{figure}[h!]
    \centering
        
    \begin{lstlisting}[frame=single, numbers=none]
"Short Trustword Mapping (S-TWM) requires a number of 
Trustwords that MUST retain at least 64 bits of entropy. 
Thus, S-TWM results into at least four Trustwords to be 
compared by the user."
    \end{lstlisting}

    \caption{Most recent RFC security recommendation}
    \label{fig:trustwordsNum}
\end{figure}
\end{center}
By sampling the \pep documentation, they have defined the use case of the Trustword handshake: \textit{``A handshake is done by comparing the Trustwords between two users through a separate communication channel (e.g. in person or by phone)''.}\footnote{\url{https://www.pep.security/docs/general\_information.html\#handshake}}. It can be assumed due to the increased globalisation of the planet that the most common handshake occurrence is over the phone, and, therefore, not in person. This is the assumed context of the handshake. This means, to create near-matches, the similarity of words is quantified phonetically instead of visually.

\section{Research Questions}

With the previously discussed points in mind, the following research questions have been proposed.

\begin{enumerate}
    \item  What are the best performing schemes to quantify phonetic similarity?  \label{goal:phoneticSimilarity}

    \item What is the time and computation complexity required to generate a 'similar' keys for a targeted key pair? \label{goal:complexity}
    
    \item What percentage of attacks successfully deceive a user? \label{goal:attackPercentage}

    \item Is the recommended minimum number of four Trustwords enough to provide a basic level of security? \label{goal:numberOfTrustwords}
\end{enumerate}
