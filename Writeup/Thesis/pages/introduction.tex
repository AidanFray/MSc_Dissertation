\chapter{Introduction}
\label{cha:Introduction}

The increasing use of public-key cryptography by instant messaging and secure email means key-fingerprint  verification is an ever more important task. 

One of the most significant risks to the security of the communication channel is a Man-in-the-middle (MiTM) attack. A MiTM attack involves an attacker impersonating one or both sides of a connection. MiTM attacks can entirely circumvent the encryption as it allows an attacker to read all of the encrypted data. 

A countermeasure for the threat of MiTM attacks is the verification of each parties’ fingerprint. A fingerprint is a small string of characters that is unique to each key and, thus, can be used to identify.

Fingerprints can come in several different encodings such as Hexadecimal, words, and even procedurally generated avatars. Previous research has shown that the average human can only hold around 7-digits ($\pm 2$) worth of data in their working memory\cite{miller1956magical}. Consequently, this makes the designing of user-friendly schemes a task of utmost importance.

Humans are commonly considered the most vulnerable part of any computer system. Solutions, therefore, have been proposed to remove this manual verification with examples such as PGP's web-of-trust\cite{callas1998openpgp} and Namecoin\cite{kalodner2015empirical}. However, these suffer from user adoption due to perceived complexity. Manual verification is, consequently, left in a difficult position, due in part to proposed solutions needing to sacrifice either usability or security. Therefore, research into improving or creating a more secure and user-friendly fingerprint encoding remains an important task.

This report evaluates the security and feasibility of attacking a fingerprint encoding known as “Trustwords.” The motivation for this is due in part to the shortage of justification behind some fundamental design choices.

The report is structured as follows:

\textbf{Chapter 2} provides any required background knowledge alongside a review of all relevant literature.

\textbf{Chapter 3} formally defines the aim of this report alongside the research questions the project aims to answer.

\textbf{Chapter 4} documents the proposed designs required to answer the research questions proposed in Chapter 3.

\textbf{Chapter 5} discusses the significant implementation details of the designs formalized in Chapter 4.

\textbf{Chapter 6} presents and analyses all experimentation results of the project.

\textbf{Chapter 7} evaluates the success of the project alongside a discussion into proposed further work. The research aims proposed earlier in Chapter 3 are used as evaluative metrics.

\textbf{Chapter 8} concludes the project with a culmination of the findings of the paper alongside any relevant conclusions.