\chapter{Conclusion}
\label{cha:conclusion}

Overall, the project has demonstrated a potential attack on \pep's implementation of Trustwords. An attack was proposed that utilised phonetic similarity algorithms to exploit the weaknesses in the Trustword dictionary to generate near-collision keys. To achieve this, a tool was created to compute a massive number of keys concurrently. It was based on a known tool but improved substantially on its key searching performance. The performance of similarity metrics was compared alongside an experiment assessing participants fallibility to the proposed attack. Results showed as much as a 32.05\% success rate for the best attack. The main project aim was to show that the recommend minimum number of four Trustwords was insufficient to provide a basic level of security. We believe we have demonstrated that four Trustwords is too low to provide enough security for general use-case and the minimum provided words should, therefore, be increased.

% \todo{Recommendation for metric - Metaphone because best balance of number of possibilities and success rate}