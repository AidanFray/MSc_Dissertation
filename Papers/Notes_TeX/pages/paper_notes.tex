\textbf{TODO:} Need to find papers on study of words are a comparative measure.

\textbf{OVERVIEW} 
\begin{itemize}
    \item Lots of research performed on E2E applications (Telegram, WhatsApp etc)
    \item Lots of study into overall representations of fingerprints
\end{itemize}



\section{Can Unicorns Help Users Compare Crypto Key Fingerprints}

This paper tested 8 different fingerprint representations with 661 different participants. All of these representations are testing using compare-and-confirm (basic comparison process for fingerprints) and compare-and-select (select from a list)

\begin{itemize}
    \item Hexadecimal
    \item Alternating vowel/consonant
    \item Words
    \item Numbers
    \item Sentences
    \item OpenSSH visual host key
    \item Vash
    \item Unicorns
\end{itemize}

The paper also emphasises the attention that has been taken to realism in the experimentation stage.

Talks about long term usability issues with public-key crypto and states fingerprints are a main source of the problem. ([28])

There seems to be mixed findings with compare-and-select and compare-and-confirm ([14, 17])

There has been research into OpenSSH fingerprints and how users interact ([19, 24])

Paper chose to target a security level of 160-bits

Papers attack strengths were $2^{40}$, $2^{60}$ and $2^{80}$

Paper states that MTurk users have been shown to be younger and better educated than the general US population ([15])

\subsection{Findings}

\begin{itemize}
    \item Graphical representations have mixed success rates (with quick comparison times)
    \item Recommendations not to use \textbf{compare-and-select}
    \item \textbf{No method} really provided enough security for high risk situations, they recommend removing users from the loop by using smart phone cameras etc
\end{itemize}


\section{An Empirical Study of Textual Key-Fingerprint Representations}

Study involved 1047 participants evaluating 6 different textual representations on MTurk. The study also includes an evaluation into usability.

The study has an attacker power of around 80 of 112 bits.

\begin{itemize}
    \item X
\end{itemize}

More mention to decentralised method finding it hard to find adoption such as Web of Trust and Namecoin ([7, 13, 30])

References as system called CONIKS and others ([24, 39, 27]) that aim to provided a directory of keys

States that many systems still rely on fingerprint comparison ([17])

States that the \textbf{hexadecimal} encoding is used in most systems. 

States again that fingerprint comparison is seldom done in practice ([17, 37])

Studies have shown that users find it difficult to compare long and "meaningless" strings ([19])

References to other word lists used. PGP Word list [22] and english word list compiled by K.C. Ogden [31]. The Peerio word list is generated from the most common words in english subtitles.

Interesting recommendation into the use of \textit{scrypt} [34] and \textit{Argon 2} [3] can be used to shorten the fingerprint length. These work slowly and a prevent easy computation of pre-images.

When generating partial-collision fixing the bits at the start and end has been shown a best practice ([17, 37])

\subsection{Findings}
\begin{itemize}
    \item Overall recommendation is to replace \textbf{hexadecimal} with \textbf{sentence} based encoding
    \item Findings show that \textbf{hexadecimal} and \textbf{Base32} perform worse than the other alternatives in realistic threat models. With over \textbf{10\%} of users failing to detect attacks with these schemes.
\end{itemize}