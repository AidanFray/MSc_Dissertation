\textbf{TODO:} Need to find papers on study of words are a comparative measure.

Lots of research has been done into the secure pairing of devices with minimal human involvement:

\begin{itemize}
    \item Using Camera Phones for Human-Verifiable Authentication
    \item Loud and Clear: Human-Verifiable Authentication Based on Audio
    \item A human-verifiable authentication protocol using visible laser light
    \item Secure Device Pairing based on a Visual Channel
    \item NFC
    \item Physical contact
\end{itemize}

\textbf{OVERVIEW} 
\begin{itemize}
    \item Lots of research performed on E2E applications (Telegram, WhatsApp etc)
    \item Lots of study into overall representations of fingerprints
\end{itemize}

\section{Can Unicorns Help Users Compare Crypto Key Fingerprints}

This paper tested 8 different fingerprint representations with 661 different participants. All of these representations are testing using compare-and-confirm (basic comparison process for fingerprints) and compare-and-select (select from a list)

\begin{itemize}
    \item Hexadecimal
    \item Alternating vowel/consonant
    \item Words
    \item Numbers
    \item Sentences
    \item OpenSSH visual host key
    \item Vash
    \item Unicorns
\end{itemize}

The paper also emphasises the attention that has been taken to realism in the experimentation stage.

Talks about long term usability issues with public-key crypto and states fingerprints are a main source of the problem. ([28])

There seems to be mixed findings with compare-and-select and compare-and-confirm ([14, 17])

There has been research into OpenSSH fingerprints and how users interact ([19, 24])

Paper chose to target a security level of 160-bits

Papers attack strengths were $2^{40}$, $2^{60}$ and $2^{80}$

Paper states that MTurk users have been shown to be younger and better educated than the general US population ([15])

\subsection{Findings}

\begin{itemize}
    \item Graphical representations have mixed success rates (with quick comparison times)
    \item Recommendations not to use \textbf{compare-and-select}
    \item \textbf{No method} really provided enough security for high risk situations, they recommend removing users from the loop by using smart phone cameras etc
\end{itemize}

\newpage
% \section{An Empirical Study of Textual Key-Fingerprint Representations}

% Study involved 1047 participants evaluating 6 different textual representations on MTurk. The study also includes an evaluation into usability.

% The study has an attacker power of around 80 of 112 bits.

% \begin{itemize}
%     \item Alphanumerical
%     \item Numerical
%     \item Words
%     \item Sentences
% \end{itemize}

% More mention to decentralised method finding it hard to find adoption such as Web of Trust and Namecoin ([7, 13, 30])

% References as system called CONIKS and others ([24, 39, 27]) that aim to provided a directory of keys

% States that many systems still rely on fingerprint comparison ([17])

% States that the \textbf{hexadecimal} encoding is used in most systems. 

% States again that fingerprint comparison is seldom done in practice ([17, 37])

% Studies have shown that users find it difficult to compare long and "meaningless" strings ([19])

% References to other word lists used. PGP Word list [22] and english word list compiled by K.C. Ogden [31]. The Peerio word list is generated from the most common words in english subtitles.

% Interesting recommendation into the use of \textit{scrypt} [34] and \textit{Argon 2} [3] can be used to shorten the fingerprint length. These work slowly and a prevent easy computation of pre-images.

% When generating partial-collision fixing the bits at the start and end has been shown a best practice ([17, 37])

% \subsection{Findings}
% \begin{itemize}
%     \item Overall recommendation is to replace \textbf{hexadecimal} with \textbf{sentence} based encoding
%     \item Findings show that \textbf{hexadecimal} and \textbf{Base32} perform worse than the other alternatives in realistic threat models. With over \textbf{10\%} of users failing to detect attacks with these schemes.
%     \item According to their study, \textbf{words} and \textbf{sentences} we some of the best methods for avoiding attack
%     \item Words has the second highest usability ratings second to only sentences
% \end{itemize}

% \newpage
% \section{Usability and Security of Out-Of-Band Channels in Secure Device Pairing Protocols}

% Paper looks into the positives of secondary channels used to authenticate device pairing. Such as device fingerprint comparison. It also aims to look into the fact that many studies performed do not take into consideration the aspect of human interaction.

% Total of 30 paid participants were recruited where the experiment simulated a P2P payment system.

% They collected:
% \begin{itemize}
%     \item Time to completed the associated process
%     \item Number of security or non-security related errors
% \end{itemize}

% And quantified using questionnaires:
% \begin{itemize}
%     \item Ease of use with the representation
%     \item Satisfaction with time spent with a method/representation
%     \item Participant's confidence with the scheme
% \end{itemize}

% The paper also tested methods:
% \begin{itemize}
%     \item Compare and Confirm
%     \item Compare and Select
%     \item Compare and Enter
%     \item Barcode scanning
% \end{itemize}

% Representations:
% \begin{itemize}
%     \item Numerical
%     \item Alphanumerical
%     \item Words
%     \begin{itemize}
%         \item Dictionary of 1024 words each mapped to 10-bits
%     \end{itemize}
%     \item Sentences
%     \item Images
%     \item Melodies
%     \item Sound (Numerical/Alphanumerical)
% \end{itemize}

% "People are the weakest link in the security chain" [26]

% Mentions a normal channel as a "Dolev-Yao" channel [5]. This is where an attacker can overhead, delete and modify messages.

% Talks about systems that were secure on paper that ended up in practice being miss-handled to reduce security. One example is with the Russian army in WWI [1]

% Bruce Schneier [27] also argues that systems and involves humans therefore often systems are broken due to improper use.


% \subsection{Findings}
% \begin{itemize}
%     \item Showed for \textbf{usability} methods ranked:
%     \begin{itemize}
%         \item Comparing-confirm
%         \item Typing strings
%         \item Comparing-select
%         \item Barcode
%     \end{itemize}

%     \item Showed for \textbf{combined security} and \textbf{usability} methods ranked:
%     \begin{itemize}
%         \item Typing strings
%         \item Barcodes
%         \item Comparing-confirm
%         \item Comparing-select
%     \end{itemize}

%     Showed that the best methods in terms of the SUM score [24] ranked the top 3 as:
%     \begin{itemize}
%         \item Numeric(C\&C)          [73.7]
%         \item Alphanumerical(C\&C)   [72.5]
%         \item Words (C\&C)           [70.6]
%     \end{itemize}

%     Numerical had 0\% security failures, Word 3.3\% and Alphanumerical had a high 13.3\% 

%     NOTE: This is a very small sample size.
% \end{itemize}

% \newpage
% \section{SafeSlinger: An Easy-to-use and Secure Approach for
% Human Trust Establishment}

% The aim of SafeSlinger is to solve the issue of key exchange. They aim to leverage the initial personal contact of people to facilitate trust. To solve the issue of people that would never meet they implement a system of "shared acquaintance"

% Talks about issues with decentralised (PGP [42]) and centralised (SSL CA [27, 34]) key exchange.

% "Tim Berners-Lee has called upon security researchers and professionals to design a public key encryption system for the people [11]"

% Only uses \textbf{24-bits} of a SHA-1 hash of all the exchanged information, \textbf{could this be exploited?}

% \newpage
% \section{A Study of User-Friendly Hash Comparison Schemes}

% The aim of this study is to provide a study on what hash comparison scheme provides the best accuracy and shortest comparison time.

% The paper also proposes new schemes: An extension to Flag and three asian character representations. 

% An interesting aspect is the effort put into the study to link characteristics about the participants, i.e. gender or age to how well they perform or rate the usability of the scheme. In total they had 436 participants.

% Overall they compare:
% \begin{itemize}
%     \item Base32
%     \item English words
%     \item Random Art
%     \item Flag
%     \item T-Flag
%     \item Flag ext
%     \item Chinese, Korean, Japanese characters
% \end{itemize}

% Key phrase: "We did not use hexadecimal digits because they're similar to Base32 and known to be error prone" [NO REFERENCE]. This is shown in \textit{``An Empirical Study of Textual Key-Fingerprint Representations''} that this isn't true - ``However, our work shows that numeric representations actually perform significantly better than Base32 and is less error prone.''

% They study also considers the comparison of "easy" and "hard" pairs of hashes. Hard is where the hashes are designed to be very similar. This is designed to provide a base-line (easy) and a worst-case (hard).

% The paper deals with ranges of entropy of 22-28bits.

% Another system that uses word is the Unmanaged Internet Architecture [4]

% States that users make errors when comparing long strings [11]

% Dicussions are also made into the pre-requirements needed for some of these schemes. Random-Art needs a colour display and requires a relativly high computational cost to generate a representation. Asian characters require unicode or codec support.

% Paper also touches on ``Additional Benefits'' - the paper talks about the ability to verbally describe the scheme, this is so the encoding scheme can be verified using a channel that does not involve a visual feed such as a telephone conversation

% Overall the paper provides a table of requirements for the schemes with the graphical representations requiring a large number of pre-requirements.

% \subsection{Findings}
% \begin{itemize}
%     \item Source, age and gender have no significant impact on the accuracy across the schemes. But younger participants were significantly faster
%     \item In general, Base32, Random Art, T-Flag and Flag Ext provide fast and accurate comparisons
%     \item Comprehension of a language when used as a encoding scheme affects the ability to distinguish hard pairs by a significant amount
% \end{itemize}

% \newpage
% \section{Usability Analysis of Secure Pairing Methods}

% Paper is aiming to perform a comparative usability evaluation of \textbf{selection methods} of hash comparisons

% This paper is doing this study in the context of secure device pairing in a diffie-hellman type key exchange. However, the results of the study could directly apply to that of hash comparison, however, it must be considered that the comparisons here are for \textbf{very} short strings. Around 4-digits 

% Comparison methods compared:
% \begin{itemize}
%     \item Compare-and-Confirm
%     \item Select-and-Confirm
%     \item Copy-and-Confirm
% \end{itemize}

% The paper also compared to methods of passphrase comparison that weren't relevant.

% Paper also included demographical information in the study, with two round of 40 participants in the usability studies.


% Findings:

% Round 1:
% \begin{itemize}
%     \item Copy-and-confirm was percieved as hard to use
%     \item Copy-and-confirm has the lowest fatal error rate but the longest comparison time and lowest percieved usability
%     \item The schemes reduced fatal error rate as their comparison times rose
%     \item Copy-and-confirm had issues with user confusion in terms of use
% \end{itemize}

% In the second round they only decided to continue with

% \begin{itemize}
%     \item Compare-and-Confirm
%     \item Select-and-Confirm
% \end{itemize}

% They dropped Copy-and-confirm due to its high similarity to the other scheme "Copy" that was being used for passphrase sharing

% The second round included some UI changes that were backed up by external research.

% This round the changes performed on the schemes resulted in much lower fatal error rates. However, Select-and-confirm had an unacceptable 5\% error rate. Compare-and-confirm had 0\% fatal error rate. The was an improvment from 20\% to 0\%.

% Both schemes were again percieved as easy to use

% Overall it was shown that compare-and-confirm was the superior choice.

\newpage
\section{On the Pitfalls of End-to-End Encrypted Communications: A Study of Remote Key-Fingerprint Verification}

Paper compares the fingerprint verification task in the context of Instant messaging E2E The paper is also choosing to focus on the remote vs proximity ways of comparing fingerprints. They aim to study the security and usability of human-centred E2E key verification.
The study ran with 25 participants.

Seems to be the first study of remote code verification study

Study compares Numerical, Image and verbally spoken codes and QR just for proximity alongside all the other alternatives.

The studies main aim to compare remote to proximity, not compare the various methods

The research is split into two sub-goals: Robustness - looks at the FAR and FRR. User experience and perception - System usability, Comfort, Satisfaction and Adaptability

States that numerical is the one of the most common representations, but out of the chosen methods 50\% use numerical and the rest hexadecimal?

They included 3 level of incorrect codes. One char change, one block and the entire fingerprint

Overall Audio was one of the most secure and easiest to use. However results for image based comparisons were not controlled in any way due to the author being unable to access the telegram image generation, this means while the other schemes had controlled restraints the image did not. Images was rated the worst.


Findings:
\begin{itemize}
    \item High False Acceptance Rate (False Negative) for all code verification methods 
    \item Results point to low usability in remote code verification settings, aside from audio based verification ever scheme has above 20\% False Rejection Rate (False Positive)
    \item In proximity settings users were able to detect attacks with minimal error rates
\end{itemize}

\newpage

\section{The drunken bishop: An analysis of the OpenSSH
fingerprint visualization algorithm}

The aim of this research is to study the OpenSSH Visual Host Key. The authors claim that the scheme was heuristically designed and has very little backing in terms of formal research, this paper aims to create a foundation for this research.

OpenSSH uses MD5 to generate a 128-bit fingerprint of the Server's key, this is then split into 64 2-bit pairs. These are then traversed byte wise in a right to left over the bit pairs. Each pair defines the direction of travel. If the snake reaches a position multiple times a pre define character is added depending on the number of visits (for example, 6 times == "B")

The paper also discusses ways to create collisions:

In conclusive results and more of an inital PoC

\begin{itemize}
    \item Brute force
    \item Graph theory
    \item Brute forcing a visual set
\end{itemize}

\newpage

\section{Improving the Robustness of Wireless Device
Pairing Using Hyphen-Delimited Numeric
Comparison}

In the context of secure device pairing the paper aims to improve the human interaction with number pairing.

Lots of research has gone into Short Authentication String (SAS) numbers ([14], [10], [4])

The research was peformed on very old devices. Nokia 6030b, very out of date and not reflective of anything around at this time.

Usability study looked into Efficiency, Robustness, General Usability

Strange format of keeping the numbers exactly the same but just placing hyphens in the middle. The authors attempted to remedy this by spacing the experiments with a days gap

Study was performed with 40 participants

Very small number of values tested only 5 overall???

Initial results show promising results for hypen delimitations etc alongside improved user impressions


\newpage

\section{Loud and Clear: Human-Verifiable Authentication Based on Audio}

Paper is in the context of secure device pairing. The paper proposes the use of Loud-and-Clear which uses text to speech to vocalise a robust english like sentence derived from a public key


"Stajano and Anderson proposed a method
for establishing keys by means of a link created through physical contact [9]."

``A number of efforts have been made to involve a human user in the secondary channel in order to manually verify/compare
keys (or hashes thereof) including [24, 11, 12] and [21].''

CHECK: Manual authentication for wireless devices.

``system represents a cryptographic hash as a series of six short (up to four-
letter) words'' - N. Haller. Rfc1760: The s/key one-time password system, 1995.

System truncates the hash and encodes into 10-bit sections.

The system is based of MadLib sentences that are used mainly as amusment for children where gaps in text are filled with nouns

The effort placed into making an order for the wordlist is similar to that of PGPfone. They effectively want to make sure that one bit change, results in a completely distinct word

The paper is not backed up by any empirical data and is just a technical description on the system. It included just a performance analysis.

\newpage

\section{Whole-Word Phonetic Distances and the PGPfone Alphabet}

PGPfones designed solution was the development of a wordless based of military NATO codes.

The authors used Moby Pronunciator database of nearly 200,000 word/pronunciation pairs.

Use-case for these findings are when verification needs to occur over a auditory line.

Mentions on potential further work in the emprical evaluation of these points


TODO: An analysis of perceptual confusions among some English consonants