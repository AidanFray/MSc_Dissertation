\section{An Empirical Study of Textual Key-Fingerprint Representations}

Study involved 1047 participants evaluating 6 different textual representations on MTurk. The study also includes an evaluation into usability.

The study has an attacker power of around 80 of 112 bits.

\begin{itemize}
    \item Alphanumerical
    \item Numerical
    \item Words
    \item Sentences
\end{itemize}

More mention to decentralised method finding it hard to find adoption such as Web of Trust and Namecoin ([7, 13, 30])

References as system called CONIKS and others ([24, 39, 27]) that aim to provided a directory of keys

States that many systems still rely on fingerprint comparison ([17])

States that the \textbf{hexadecimal} encoding is used in most systems. 

States again that fingerprint comparison is seldom done in practice ([17, 37])

Studies have shown that users find it difficult to compare long and "meaningless" strings ([19])

References to other word lists used. PGP Word list [22] and english word list compiled by K.C. Ogden [31]. The Peerio word list is generated from the most common words in english subtitles.

Interesting recommendation into the use of \textit{scrypt} [34] and \textit{Argon 2} [3] can be used to shorten the fingerprint length. These work slowly and a prevent easy computation of pre-images.

When generating partial-collision fixing the bits at the start and end has been shown a best practice ([17, 37])

\subsection{Findings}
\begin{itemize}
    \item Overall recommendation is to replace \textbf{hexadecimal} with \textbf{sentence} based encoding
    \item Findings show that \textbf{hexadecimal} and \textbf{Base32} perform worse than the other alternatives in realistic threat models. With over \textbf{10\%} of users failing to detect attacks with these schemes.
    \item According to their study, \textbf{words} and \textbf{sentences} we some of the best methods for avoiding attack
    \item Words has the second highest usability ratings second to only sentences
\end{itemize}