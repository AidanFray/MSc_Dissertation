\section{Wiretapping via Mimicry: Short Voice Imitation
Man-in-the-Middle Attacks on Crypto Phones}

The aim of the paper is to simulate a MITM attack on SAS and the voice channel used to authenticate

``Short voice reordering attack'' builds arbitrary SAS
strings in a victim’s voice by reordering previously eavesdropped SAS strings spoken by the victim.

called the ``short voice morphing attack;;, builds arbitrary SAS strings in a victim’s voice from a few previously eavesdropped sentences (less
than 3 minutes) spoken by the victim.

Is attempting to prove the assumption that "Mallory" being in control of the channel can controls Communications but cannot impersonate alice or bob.

Paper showed, that it is very possible to successfully attack SAS strings. Authors do comment on their difficulty in generating longer strings i.e. longer hash digests

Findings:
\begin{itemize}
    \item Success is more likely with shorter strings
    \item Showed that with 80\% success participants can distinguish a different voice from a familiar voice
    \item Their morphed attack was overall just above 50\% however, their reordering attack had almost 80\% success rate
    \item Both attacks were above the baseline of a different voice. I.e. the case of Mallory just replacing the audio with their voice.
\end{itemize}